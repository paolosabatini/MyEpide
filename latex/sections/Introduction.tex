\section{Introduction}
The document studies and develop the model Susceptible-Exposed-Infected-Quarantined-Recovered-Susceptible (SEIQRS) described in literature \cite{MingLiu} in a Small-World approximation. This model classifies the population into five connected categories and studies their evolution in time. Susceptible people consist of people not got in touch with the virus. The exposed category is made of people in contact with an infected person. These people are not contagious and the virus are in incubation. After an incubation time an exposed person becomes infected, bacoming contagious but without heavy symptoms. Part of the infected category evolves heavy symptoms and they are hospitalised and quarantined. Those people may recover or die. However a recovered person may not be immune, therefore are still susceptible. The mathematical description of the model is given in Section~\ref{sec:model}. The coupled equations describing evolution of the categories are introduced, following the notation in \cite{MingLiu}. The results are compared with the literature and discussed. \\

In late December a cluster of this new pneumonia virus COVID-19 has been recorded in China, specifically in Wuhan city, spreading in all the Hubei region \cite{COVID-Clinical}. Despite the strict measures adopted in this region to contain the outbreak, the virus has spread worldwide, starting from Europe. The first region strongly affected by the virus was Lombardy, in Italy, where around $1000$ people resulted infected on March 1st and currently counting more than $60000$ cases \cite{Lab24}. Currently, on April 15th 2020, the most affected area is USA, while Spain is the country counting the largest amount of infected people in Europe \cite{WHO}. More than two million infected people are in the world, counting, affecting more than $200$ different countries with more than $100000$ confirmed deaths \cite{WHO}. The data of the evolution of the outbreak in Italy and Lombardy are reported in Section~\ref{sec:data}.\\

The dataset corresponding to Lombardy outbreak is used to test the performance of the model \cite{Lab24}. The model has been adapted to fit the adopted social-distancing and testing campaigns. The parameters of the mathematical model are measured against the data through a Profile Likelihood Fit (PLF) and used to estimate the out break evolution. This is described in Section~\ref{sec:covid}.\\

Conclusions and further considerations are discussed in Section~\ref{sec:conclusions}.

