\section{Impact of parameters on model predictions}
In this subsection the impact of the parameters on the model predictions is analysed. The most important parameters for the evolution of the infected population and the occurrence of the outbreak are $\beta$ and $\braket{k}$ parameters. The $\beta$ parameters is connected to the infection factor, so the probability of infecting a susceptible person in contact with and infected one. The $\braket{k}$ parameters indicate instead the number of connection for each individual, so measures the level of interaction of the small world. Since the two parameters always come together their effect is the same on the evolution and visible in Figure~\ref{fig:scan_i_vs_beta}, the larger those parameters are the steeper is the curve of the infected population, as expected from Equation~\ref{eqn:idot}.  A large impact on the outbreak evolution is also given by the fraction of quarantined infected population $\delta$, as shown in Figures~\ref{fig:scan_i_vs_delta} and \ref{fig:scan_q_vs_delta}. The outbreak is significantly contained in case a larger fraction of infected population is spotted and quarantined.

\begin{figure}[!ht]\centering
\subfloat[\label{fig:scan_i_vs_beta}]{\includegraphics[width=0.32\textwidth]{imgs/ModelDescription/Scan_I_vs_beta_parameters_alternative.pdf}}
\subfloat[\label{fig:scan_i_vs_delta}]{\includegraphics[width=0.32\textwidth]{imgs/ModelDescription/Scan_I_vs_delta_parameters_alternative.pdf}}
\subfloat[\label{fig:scan_q_vs_delta}]{\includegraphics[width=0.32\textwidth]{imgs/ModelDescription/Scan_Q_vs_delta_parameters_alternative.pdf}}
\caption{Effect on the evolution of the infected population given by varying the $\beta$ parameter (a). Effect on infected (b) and quarantined (c) population given by varying the $\delta$ parameter.}.
\end{figure}

Other effects are given by $\gamma$, $\mu$ and $d_{1/2}$ parameters. Those control more the recovery curve. The $\gamma$ parameter is related to the probability of a recovered individual to not develop immunity: as shown in Figure~\ref{fig:scan_r_vs_gamma}, the recovered people has a decrease after the peak of the outbreak. The recovery curve is strongly affected by the $\mu$ parameter that regulates the recovery probability of the infected population: as shown in Figure~\ref{fig:scan_r_vs_mu}. The mortality $d_{1/2}$ instead increases the number of deaths, given by the difference of the recovered population and the initial $10000$ susceptible people, shown in Figure~\ref{fig:scan_r_vs_d1}.

\begin{figure}[!ht]\centering
\subfloat[\label{fig:scan_r_vs_gamma}]{\includegraphics[width=0.32\textwidth]{imgs/ModelDescription/Scan_R_vs_gamma_parameters_alternative.pdf}}
\subfloat[\label{fig:scan_r_vs_mu}]{\includegraphics[width=0.32\textwidth]{imgs/ModelDescription/Scan_R_vs_mu_parameters_alternative.pdf}}
\subfloat[\label{fig:scan_r_vs_d1}]{\includegraphics[width=0.32\textwidth]{imgs/ModelDescription/Scan_R_vs_d1_parameters_alternative.pdf}}
\caption{Effect on the evolution of the recovered population given by varying the $\gamma$ (a), $\mu$ (b) and $d_1$ (c) parameters.}.
\end{figure}

The only temporal parameter is the incubation time $\tau$: it also strongly affects the evolution of the oubreak, shown in Figure~\ref{fig:scan_i_vs_tau}. In case of long incubation time, the outbreak develops a  multiple peak structure, while in case of a short incubation time a clean gaussian peak is recognised.

\begin{figure}[!ht]\centering
\includegraphics[width=0.4\textwidth]{imgs/ModelDescription/Scan_I_vs_tau_parameters_alternative.pdf}
\caption{Effect on the evolution of the infected population given by varying the $\tau$ parameter.}
\label{fig:scan_i_vs_tau}
\end{figure}
