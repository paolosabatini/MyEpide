\documentclass{article}
\usepackage[T1]{fontenc}
\usepackage[utf8]{inputenc}
\usepackage{titlesec}
\usepackage{braket}
\usepackage{booktabs} 
\usepackage[margin=1in]{geometry}
\usepackage{graphicx}
\usepackage{subfig}
\usepackage{xcolor}
\usepackage{float}
\usepackage{hyperref}

%%% 
% Define new style of sections
%%%
\titleformat*{\section}{\LARGE\bfseries}
\titleformat*{\subsection}{\Large\bfseries}
\titleformat*{\subsubsection}{\large\bfseries}
\titleformat*{\paragraph}{\large\bfseries}
\titleformat*{\subparagraph}{\large\bfseries}
%%%
% Title page
%%%
\title{
\begin{flushleft}
\rule{\textwidth}{1pt}\\
  \textsc{\textbf{Elementary study of outbreak evolution adapted to the covid-19 virus example 	}}\\[2mm]
\textsc{\large Paolo Sabatini}\\
\rule{\textwidth}{1pt}
  \end{flushleft}
}
%\author{
%\begin{flushleft} 
%\textsc{Paolo Sabatini} 
%\end{flushleft} 
%}

\date{}


 
\begin{document}

\maketitle


\begin{abstract}
This document collects a brief study of modeling an outbreak. The model framework is Susceptible-Exposed-Infected-Quarantined-Recovered-Susceptible (SEIQRS), present already in litealready in past publicationsrature. Here the model is independently re-implemented, studied and analysed. The effects of the model parameters are checked. The model is tested against the national- and regional-wide Italian COVID-19 outbreak data. A fit against the data is performed to measure the model parameters and to give predictions of the expected outbreak behaviour.
\end{abstract}
\vspace{2cm}
\tableofcontents

\newpage

\section{Introduction}
The document studies and develop the model Susceptible-Exposed-Infected-Quarantined-Recovered-Susceptible (SEIQRS) described in literature \cite{MingLiu} in a small-world approximation. This model divides the available population into these five categories and make them evolve in time. All the individuals are connected to a subset of the population, regulated via dedicated parameter. The model is described and studied in Section~\ref{sec:model}. In this section the coupled equations for the population evolution are introduced, following the notation in \cite{MingLiu}, and the obtained results are compared with the literature and discussed. The impact on the model predictions of the parameters is checked. The collected data in Italy during the COVID-19 outbreak in Lombardy and Italy are reported in Section~\ref{}, and used as benchmark to refine the model and provide a description of the outbreak evolution.
Templates of the total cases distributions in time are extracted for some parameters choice and used to perform a Template fit on the Lombardy dataset. This is shown in Section~\ref{}.
\section{Description of the model}
\label{sec:model}

In the following subsection the mathematical implant is shown, followed by a discussion of the obtained results compared to literature. Finally, the effects of the parameters are understood and presented.

\subsection{Mathematical background}
The notation and the model description has been already presented in \cite{MingLiu}, here a brief overview, necessary to understand the document, is given. The model framework is SEIQRS: the population of $N$ individuals are separated into these five classes. Quantities with capital letters (such as $S$, $E$\dots) indicate the real number of individuals belonging to the given class, whether quantities given wih small letters (such as $s$, $e$\dots) correspond to fractions (i.e. $s = S/N$). \\

The suscpetible $S$ class correspond to all people who can be potentially infected. In time, they decrease by the number of people getting exposed (thus in touch with an infected individual), and increased by the number of recovered people not developing immunity after being infected. The time variation of $\dot{s} = \frac{ds}{dt}$ is then:
\begin{equation}
\dot{s} (t)= -\beta\braket{k}s(t)i(t)+\gamma{r(t)} 
\end{equation} 
The exposed $E$ people are individuals got in touch with an infected. An exposed person, after $\tau$ days, they turn to infected. Then the equation for $\dot{e}$:
\begin{equation}
\dot{e} (t)= \beta\braket{k}s(t)i(t) - \beta\braket{k}s(t-\tau)i(t-\tau) 
\end{equation} 
Infected people $I$ decrease by the number of deaths and quarantined people, giving the equation for $\dot{i}$ :
\begin{equation}
\dot{i} (t)= \beta\braket{k}s(t-\tau)i(t-\tau) - d_1i(t) - \delta {i(t)}
\end{equation} 
Quarantined people $Q$ may die or recover, therefore $\dot{q}$ follows the equation:
\begin{equation}
\dot{q} = \delta{i(t)} - d_2{i(t)} - \mu{i(t)}
\end{equation}
Finally, recovered people may actually turn back in susceptible, then the equation for $\dot{r}$:
\begin{equation}
\dot{r} = \mu{i(t)}-\gamma{r(t)}
\end{equation}

The small-world assumption makes all individuals interact with each other, and the numbers of interactions per day is measured by the $\braket{k}$ parameter. The initial conditions state $i(0) = i_0 \ll 1$, $e(0) = e_0 = \braket{k}i_0$ and $s(0) = s_0 = 1-i_0-e_0$. The coupled equations are solved with a second-order Runge-Kutta Method. No substantial discrepancy has been observed by using Euler method.


\subsection{Comparison with literature}
The results obtained are cross-checked with the literature \cite{MingLiu}.
\section{Dataset}

The dataset used to test the predictability of the developed model are about the COVID-19 oubreak in Italy. This dataset is used because the quarantine procedure has been implemented quite at the beginning of the outbreak, as in the model. Moreover Italy is a high-density country, especially in the Lombardy area, so that the small-world approximation may be feasible to describe the oubreak. Data are extracted from the web-site of \emph{Il Sole 24 Ore} \cite{Lab24}, the main economical italian newspaper, that has been one of the first sources of data divided by regions. Data in Italy and Lombardy are shown in Figures~\ref{fig:data_italy} and \ref{fig:data_lombardy}.\\

\begin{figure}[!ht]\centering
\subfloat[\label{fig:data_italy}]{\includegraphics[width=0.4\textwidth]{imgs/Dataset/Data_Italy.pdf}}
\subfloat[\label{fig:data_lombardy}]{\includegraphics[width=0.4\textwidth]{imgs/Dataset/Data_Lombardia.pdf}}

\caption{Dataset collected in Italy (a) and Lombardy (b).}.
\end{figure}

A focus on Lombardy dataset is given, since it is by far the most affected region in Italy, completely driving the numbers of the outbreak in the whole country, and it is smaller, high-populated, very dynamic and connected area, where small-world approximation may work better.
\section{Application to the COVID-19 outbreak}

The model described in Section~\ref{sec:model} is tuned here to fit at best the dataset from Lombardy shown in Figure~\ref{fig:data_lombardy}. Since the data available show just the leading curve of the outbreak, only parameters $\beta$, $\braket{k}$, $\delta$ and $\tau$ have to be tuned. Moreover, an additional parameter $t_0$ has been added to shift the distribution in time. The available dataset regards the \emph{total cases}, corresponding to the sum of infected (positively tested), recovered and deaths. This is not immediately comparable with the population classes in the model. Here it is identified as the sum of quarantined, recovered and deaths. Infected population is not included in the total number because it would correspond to the whole infected population being tested as positive, even the population not showing symptoms, highly probable in young population. Therefore the infected population correspond to the part of the population that has the virus but has not been tested, and all the positively tested population is immediately quarantined. The best fitting setup is shown in Figure~\ref{fig:data_vs_model_first_lombardy}. Parameters modifying the trailing edge of the infection is set to approximate values of the outbreak. Mortality is set to $2\%$, reinfection probability to $10^{-5}$ since yet no cases have been observed. Population is set to $10^7$, corresponding to Lombardy population.\\

\begin{figure}
\centering
  \includegraphics[width=0.4\textwidth]{imgs/Covid/DataVsModel_parameters_Lombardia_less_impacting.pdf}
  \caption{Best match of parameters to make the model described in Section~\ref{sec:model} fit with data il Lombardy.}
  \label{fig:data_vs_model_first_lombardy}
\end{figure}

As shown in Figure~\ref{fig:data_vs_model_first_lombardy} the model predicts an exponential growth until reaching a peak, that is not what is measured, since different slopes in the total cases curves are present. Possibly this is due to changing in the small-world conditions such as limitation to the social activity (mandatory closings of shops, cancellations of events, etc.), improvements of the sanitary checking and testing. This should cause a decrease of the slope, instead of an increase, caused, for example, by a longer incubation time. This hypothesis is quite encouraged by the dates on data when the change of slopes occur: around March 10th and 20th. On March 9th and 11th two special laws have been approved and applied, closing all shops, limitation of movements for citizens. On March 10th even a tighter regulation have been introduced: no movements among cities and mandatory closing of all the activities except for hospital, farmacies, hostera and corresponding chains. The corona virus' incubation time, although the recommended quarantine of fifteen days, it has a shorter incubation time before possible symptoms happen of less than a week. Therefore the effects of the actions affect the curve in few days.\\

The idea to adapt the model to the real world scenario, two improvements have been implemented:
\begin{itemize}
\item Scheduling of $\delta$ parameter: this reflects the different sanitary and testing coverage versus time.
\item Scheduling of $\braket{k}$ parameter: this reflects the change in social activiti in time.
\item {color{red}Fluctuation of $\delta$ parameter: this reflects the fluctuation of the number of tests done per day.}
\end{itemize}


\section{Improvement of the model}
Here the scheduling of the parameters is described. The $\delta$ parameter regulates the fraction of infected people that is positively tested and quarantined. The tests and quarantine pressure has been increased in time, following a trend similar to Figure~\ref{fig:scheduling}. This estimation is based on the trend followed by the total number of tests performed against time, passing from $10\%$ to $100\%$ of the current number in ten days. The maximum cover is given by the set $\delta$ parameter in input to the simulation. The $\braket{k}$ scheduling represents the limitation of the social interactions, from a freely-communicating system to isolation where an infected person may interact only with a susceptible individual.

\begin{figure}
\centering
  \includegraphics[width=0.4\textwidth]{imgs/Covid/Scheduling.pdf}
  \caption{Scheduling of $\delta$ and $\braket{k}$ parameters in time. The reference values of the parameters are set the same as in Figure~\ref{fig:data_vs_model_first_lombardy}.}
  \label{fig:scheduling}
\end{figure}



\begin{thebibliography}{9}
\bibitem{MingLiu} 
Z.~Gao, M.~Liu and Y.~Xiao, 
\textit{Modeling and Analysis of Epidemic Diffusion within Small-World Network},
Journal of Applied Mathematics (2012) 841531.

\bibitem{MingLiuOld} 
M.~Liu and L.~Zhao, 
\textit{Analysis for epidemic diffusion and emergency demand in an anti-bioterrorism
system},
International Journal of Mathematical Modelling and Numerical Optimisation (2011), vol. 2, no. 1, pp. 51–68.

\bibitem{Lab24}
Il Sole 24 Ore,
\textit{ 
Coronavirus in Italia, i dati e la mappa},
\href{https://lab24.ilsole24ore.com/coronavirus/}{https://lab24.ilsole24ore.com/coronavirus/}

\bibitem{ROOT}
I.Antcheva et al., 
\textit{ 
ROOT — A C++ framework for petabyte data storage, statistical analysis and visualization},
Computer Physics Communications (2009), vol. 180, no. 12.

\bibitem{HistFactory}
Cranmer, Kyle and Lewis, George and Moneta, Lorenzo and Shibata, Akira and Verkerke, Wouter,
\textit{ 
HistFactory: A tool for creating statistical models for use with RooFit and RooStats},
CERN-OPEN-2012-016.


\end{thebibliography}

\end{document}
